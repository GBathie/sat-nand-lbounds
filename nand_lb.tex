\documentclass{article}
\usepackage[utf8]{inputenc}
\usepackage[T1]{fontenc}
\usepackage[english]{babel}
\usepackage{amsfonts, amsmath, amsthm, amssymb}
\usepackage[margin=2.5cm]{geometry}
\usepackage{enumitem}

\title{NAND formulas lower bounds}

\newcommand{\NN}{\mathbb{N}}%
\newcommand{\Sbb}{\mathbb{S}}%
\newcommand{\Cc}{\mathcal{C}}%
\newcommand{\Gg}{\mathcal{G}}%
\newcommand{\Ss}{\mathcal{S}}%
\newcommand{\Ot}{\tilde{O}}%
\newcommand{\Poly}{\textsf{P}}%
\newcommand{\NP}{\textsf{NP}}%
\newcommand{\bit}{\{0,1\}}%
\DeclareMathOperator*{\poly}{poly}
\DeclareMathOperator*{\polylog}{polylog}


\theoremstyle{plain}
\newtheorem{theorem}{Theorem}[section]
% \newtheorem{problem}{Problem}
\newtheorem{prop}[theorem]{Proposition}
\newtheorem{property}[theorem]{Property}
\newtheorem{lemma}[theorem]{Lemma}
\newtheorem{corollary}[theorem]{Corollary}
\theoremstyle{definition}
\newtheorem{definition}[theorem]{Definition}
\newtheorem{remark}{Remark}

\begin{document}

\maketitle

We consider formulas with gates of fan-in 2.

\section{Shallow NAND formula for every boolean function}

\begin{definition}
	The multiplexer function with parameter $n$ is
	the function defined by:
	\[MUX_{n}: \begin{cases}
	\bit^{n+2^n} \rightarrow \bit \\
	(x_1,\ldots,x_n, r_1,\ldots, r_{2^n}) \mapsto r_t, \text{ where } x_1\ldots x_n \text{ is the binary representation of } t
	\end{cases}
	\]
\end{definition}

\begin{lemma}\label{lemma:shallow-mux}
	The $MUX_{n}$ function has a NAND formula of depth $n + 2\log n + O(1)$.
\end{lemma}
\begin{proof}
	If the structure was unrestricted, we could simply build a binary tree of OR gates of depth $n$,
	and put at each leaf a formula of depth $\log(n)$ which is equivalent to $f(y) \wedge x = y$,
	where $x$ denotes the vector $x_1,\ldots,x_n$. In that case, for a given input $x$, each of these formula except at most one is equal to zero, and if there is a one, it ``flows'' up to the output gate via the OR tree.

	The idea is to do something similar, except that the result has to be propagated to the root through a OR/AND tree. We can achieve this by ``activating'' some other well-chosen formulas.
	We use a top-down approach, starting from the output OR gate. There is nothing needed to propagate the result through OR gates. For AND gate, the results comes on one wire, and there must be a 1 on the other wire.

	In mathematical terms: we associate to each node of the OR/AND tree a bit string of length $\leq n$ 
	in the following way: 
	start with the empty string at the root, append a zero when you move to a left child, and a one when you move to a right child. Leaves have bit strings of length $n$.
	Given a leaf with bit string $y$, we add below it the formula 
	$((y = x) \wedge f(y)) \vee ((y\neq= x) \wedge y fiddle x)$ TODO.
\end{proof}

\begin{corollary}\label{cor:shallow-nand}
	Every boolean function on $n$ variables has NAND formulas of depth $n + 2\log n + O(1)$.
\end{corollary}
\begin{proof}
	We can write $f(x_1,\ldots,x_n)$ as $MUX_{n}(x_1,\ldots, x_n, f(0,\ldots,0),\ldots,f(1,\ldots,1))$.
	By interpreting each $f(b_1,\ldots,b_n)$ as a constant (i.e. has a formula of constant depth), we get the desired result.
\end{proof}

\section{Andreev's function(s) have NAND formulas of size $\Ot(n^3)$}

Andreev's function with parameters $k, m$
is the function $A_{k,m} : \bit^{\bit^k} \times (\bit^m)^k \rightarrow \bit$
which maps a boolean function $f: {\bit^k} \rightarrow \bit$ 
(given as a bit string of length $2^k$ representing its truth table)
and $k$ bit strings $(X_1,\ldots, X_k)$ of length $m$
to $f(X_{11} \oplus \ldots \oplus X_{1m}, \ldots, X_{k1} \oplus \ldots \oplus X_{km})$.
$A_{k,m}$ can be seen as a boolean function which takes $n = 2^k + km$ inputs.
In can be shown that $A_{k,m}$ does not have (classical De Morgan's) formulas
of size $O(n^{3-\epsilon})$, for every $\epsilon > 0$.
This lower bound is tight: one can easily construct De Morgan formulas of $\Ot(n^3)$ size
for $A_{k,m}$.

In what follows, we show that $A_{k,m}$ also has NAND formulas of size $\Ot(n^3)$,
which shows that NAND formulas are rather expressive.

Using a simple inductive argument, one can show that any function on $n$ variables has 
NAND formulas of depth $2n$, but this construction is not sufficient to obtain the aforementioned 
formulas for Andreev's function.
Using Lemma~\ref{lemma:shallow-mux}, we can prove the following result.

\begin{corollary}
	Andreev's functions have NAND formulas of depth $3\log n + O(\log \log n)$
	(which have size $\Ot(n^3)$, since each gate has fan-in 2).
\end{corollary}
\begin{proof}
	First, notice that both $a\oplus b$ and $\neg (a\oplus b)$ have NAND formulas of depth 2.
	Therefore, the $XOR$ of $m$ variables can be computed with tree-like NAND formula of depth $2\log m + O(1)$.
	By plugging these formulas applied to each $X_i$ 
	into a NAND formula of depth $k + 2\log k + O(1)$ for $f$ given by Corollary~\ref{cor:shallow-nand},
	we get a that $A_{k,m}$ has a NAND formula of depth 
	$2\log m + k + 2\log k + O(1) \leq  3\log n + 2\log\log n + O(1)$.
\end{proof}



\section{Speedup Lemma}

\begin{lemma}\label{speedup-lemma}
	Let $F$ be a uniform OR/AND formula on $n$ variables of depth $d= c \log n$.
	There is an (co-)nondeterministic algorithm that decides
	whether an input $v$ satisfies $F$ in time $n^{c/2}\polylog(n)$.
\end{lemma}
\begin{proof}
	We first prove the nondeterministic case.
	We can assume w.l.o.g. that $d$ is even, and $F$ has depth $d$ on every path. 

	Given an OR/AND formula $F$ of depth $d$ a bit string $x$ of length at most $d$, let $F_{|x}$
	be the sub-formula defined by the subtree starting at the node reached by following the path specified by $x$ from the root or $F$.

	Consider the following recursive algorithm $A$, taking as input an OR/AND formula $F$ of depth $d > 0$, a bit string $x$, and the valuation $v$:
	\begin{itemize}
		\item 
	if $|x| < d$
	nondeterministically guess $b\in \bit$,
	and return $A(F, xb0, v) \wedge A(F, xb1, v)$.
		\item 
	if $|x| = d$, run the the algorithm $P_F$ on $x$, and return the truth value of the corresponding literal in $v$.
	\end{itemize}

	Correctness: we show that there exists a $b\in \bit$ such $A(F, x, v)$ returns true iff and only if
	the formula $F_{|x}(v)$ is true.
	This is follows by definition when $|x| = d$.
	By induction, when $|x| < d$, $A(F, xb0, v) \wedge A(F, xb1, v)$ is the value of the AND/OR sub-formula $F_{|xb}$. As $F_{|x}$ is an OR/AND formula, it is satisfiable iff at least one of $F_{|x0}$ or $F_{|x1}$ is sat, i.e. iff there exists $b\in \bit$ such that $A(F, xb0, v) \wedge A(F, xb1, v)$ is true.
	The value of $F$ is then given by $A(F, \varepsilon, v)$, where $\varepsilon$ is the empty word.

	Complexity: $A$ makes two recursive calls with the length of $x$ increased by $2$, 
	and makes a call to $P_F$ at length $d$.
	Therefore there are $2^{d/2} = n^{c/2}$ calls to $P_F$, and $A$ runs in time $n^{c/2}\polylog(n)$.
	We also see here that increasing the depth of $F$ by a constant only adds a constant factor in the runtime, hence we can assume that $F$ has even depth.

	For the co-nondeterministic case, we proceed similarly, except that we guess universally on the AND gates,
	and check both sides of OR gates.
\end{proof}

\section{Lower bounds}



\section{To keep in mind for later}
It seems that functions of the form $x_0 \vee (x_1 \wedge (x_2 \vee \ldots))$
are ``hard'', i.e. they have NAND formulas of depth at least $n$ (or $\Omega(n)$ ?).

\begin{itemize}
	\item Is this true ?
	\item Can we use it to construct a hard function like Andreev's ?
\end{itemize}


\end{document}
