\documentclass[a4paper, 11pt]{article}
% \usepackage{ae,lmodern}
\usepackage[utf8]{inputenc}
\usepackage[T1]{fontenc}
\usepackage[USenglish]{babel}
\usepackage[margin=3cm]{geometry}
\usepackage{amsfonts}
\usepackage{amsmath}
\usepackage{amsthm}
\usepackage{amssymb}
\usepackage{todonotes}
\usepackage{enumitem}
\usepackage{algorithm}
\usepackage[noend]{algpseudocode}


\theoremstyle{plain}
\newtheorem{theorem}{Theorem}[section] % Number within sec ?
% \newtheorem{theorem}{Theorem}
\newtheorem{prop}[theorem]{Proposition}
\newtheorem{property}[theorem]{Property}
\newtheorem{lemma}[theorem]{Lemma}
\newtheorem{corollary}[theorem]{Corollary}
\newtheorem{proposition}[theorem]{Proposition}
\newtheorem{exercise}[theorem]{Exercise}
\newtheorem{conjecture}[theorem]{Conjecture}
\newtheorem{observation}[theorem]{Observation}
\theoremstyle{definition}
\newtheorem{definition}[theorem]{Definition}
\newtheorem{example}[theorem]{Example}
\theoremstyle{remark}
\newtheorem{remark}[theorem]{Remark}
% \newtheorem*{remark*}{Remark}
% \theoremstyle{definition}
% \newtheorem{problem}{Problem}
% \newtheorem{openproblem}{Open Problem}

\newcommand{\NN}{\mathbb{N}}%
\newcommand{\ZZ}{\mathbb{Z}}%
\newcommand{\Sbb}{\mathbb{S}}%
\newcommand{\Cc}{\mathcal{C}}%
\newcommand{\Gg}{\mathcal{G}}%
\newcommand{\Ff}{\mathcal{F}}%
\newcommand{\Ss}{\mathcal{S}}%
\newcommand{\eps}{\varepsilon}%
\newcommand{\epsm}{\eps^{-1}}%
\newcommand{\bit}{\{0,1\}}%
\DeclareMathOperator*{\poly}{poly}

\newcommand{\Omegat}[1]{\widetilde{\Omega}\left( #1 \right)}%
\newcommand{\Ot}[1]{\widetilde{O}\left( #1 \right)}%

\newcommand{\Poly}{\textsf{P}}%
\newcommand{\Log}{\textsf{L}}%
\newcommand{\NP}{\textsf{NP}}%
\newcommand{\TIME}{\textsf{TIME}}%
\newcommand{\NTIME}{\textsf{NTIME}}%
\newcommand{\coNTIME}{\textsf{coNTIME}}%
\newcommand{\SAT}{\textsf{SAT}}%
\newcommand{\ParSAT}{\ensuremath\bigoplus\textsf{SAT}}%
\newcommand{\SharpSAT}{\#\textsf{SAT}}%
\newcommand{\NAND}{\textsf{NAND}}%
\newcommand{\ND}{\textsf{NDepth}}%
\newcommand{\NDL}[1]{\ND\left[ #1 \log n\right]}%
\newcommand{\TS}[1]{\textsf{TS}\left[ n^{ #1 }\right]}%

\newcommand{\todohere}{\todo[inline]{TODO}}%
\newcommand{\todoh}[1]{\todo[inline]{TODO: #1 }}%

%%%%%%%%%%%%%%%%%%%%%%
%%%%%%%%%%%%%%%%%%%%%%
\title{Uniform formula lower bounds for testing the existence of and counting satisfiable assignments}
\author{Gabriel Bathie, Ryan Williams}
\begin{document}
\maketitle

\begin{abstract}
\end{abstract}

%%%%%%%%%%%%%%%%%%%%%%
%%%%%%%%%%%%%%%%%%%%%%
\section{Introduction}

\todoh{Motivation: understanding complexity. Explicit lower bounds: programs, circuits, formulas.
Uniformity.}

A line of work initiated by Fortnow~\cite{fortnow2000time,fortnow2005time} in 1997 
has proven model-independent time-space lower bounds for the SAT
problem and related problems within the polynomial-time hierarchy.
This approach was then extended by Williams~\cite{williams2006inductive,williams2007time,williams2013alternation} 
into a complex framework which culminated in proving that \SAT{} 
does not have a small-space algorithm that runs in time $n^{2\cos(\pi/7)}  \simeq n^{1.801}$.
This line of work concluded~\cite{buss2015limits} with a proof 
that the $2\cos(\pi/7)$ exponent is optimal for this framework,
which implies that radically new ideas would be required to separate \NP{} from \Log{}.
This approach was later extended to obtain slightly stronger lower bounds
for quantum classes, and against small-space randomized computation~\cite{mudigonda2020time}.

In this work, we explore a new extension of this line of work, 
which gives lower bounds on the depth of uniform formulas that solve \SAT{}.
By giving a new efficient simulation of uniform log-depth \NAND{} formulas 
by $\Sigma_2$ computation, we are able to show that \SAT{}
does not have \textit{uniform} \NAND{} formulas 
of depth less than~$4 \cos(\pi/7) \log n \simeq 3.603 \log n$.

Finding strong lower bounds on the formula size of explicit functions in \NP{}
is a long-standing open problem.
The best lower bounds known depend on the computational basis, that is, 
the set of gates that are allowed in the formula.
For the full binary basis $B_2$, Nechiporuk\cite{nechiporuk66boolean}
proved in 1966 an $\Omegat{n^2}$ lower bound for an explicit function in \NP{}, 
which remains unbeaten to this day.
For the De Morgan basis $\{\wedge, \vee, \neg\}$, Andreev~\cite{andreev1987method} 
proved an $\Omegat{n^{2.5}}$ lower bound, 
which was later improved by H{\aa}stad~\cite{hastad1998shrinkage} to $\Omegat{n^{3}}$.
This lower bound is optimal up to polylog factors,
since their function has (uniform) formulas of size $\Ot{n^{3}}$.

Note that all the lower bounds mentioned in the previous paragraph 
are \textit{size} lower bounds of the form $n^{c + o(1)}$
and they hold for \textit{non-uniform} formulas.  
The lower bounds that we give are \textit{depth} lower bounds of the form $(c + o(1))\log n$,
and only hold for \textit{uniform} formulas.
While an $n^c$ size lower bound implies a $c\log n$ depth lower bounds,
the converse is not immediately true; 
one may get an $n^c$ size lower bound from an
$\alpha c\log n$ depth lower bound, where $\alpha$ depends on the computation basis,
using the work of Brent~\cite{brent1974parallel} and Spira~\cite{spira1971time}.

%%%%%%%%%%%%%%%%%%%%%%
\subsection{Results overview}

The \NAND{} basis is rather expressive: Andreev's function has \NAND{} formulas of depth $(3+o(1)) \log n$ (and therefore of size $\Ot{n^3}$), which matches the lower bound for the De Morgan Basis.
\todoh{more about nand?}

Namely, we prove the following:
\begin{theorem}\label{thm:main}
	For any $c < 4 \cos(\pi/7) \simeq 3.603$, \SAT{} does not have 
	uniform \NAND{} formulas of depth $(c + o(1)) \log n$.
\end{theorem}

%%%%%%%%%%%%%%%%%%%%%%
%%%%%%%%%%%%%%%%%%%%%%
\section{Definitions}

\subsection{Uniform formulas}
A boolean formula $\phi$ over $n$ inputs is a 
rooted full binary tree\footnote{\textit{Full} means that every node has either two or zero children.}
whose internal nodes, called gates, are labeled by a function $f: \bit^2 \rightarrow \bit$,
and leaves are labeled with a variable, $x_i$ its negation $\neg x_i$ for some $i, 1\le i \le n$,
or a constant $c\in\bit$.
The depth of a formula is the depth of the underlying tree.
Given an input $x \in\bit^n$, the value of $\phi$ at $x$ can be defined 
as the value of it root on input $x$, 
where the value of a gate is defined inductively as follows:
\[val(g) =\begin{cases}
	c & \text{ if $g$ is labeled with the constant} c\in\bit\\
	x_i & \text{ if $g$ is labeled with $x_i$}\\
	\neg x_i & \text{ if $g$ is labeled with $\neg x_i$}\\
	f(g_1, g_2) & \text{ if $g$ is labeled  $f$, and $g_1,g_2$ are children of $g$}
\end{cases}\]

In this work, we focus on \NAND{} formulas, that is formulas whose gates are all labeled 
with the function $\NAND: x,y \mapsto \neg(x\wedge y)$.
The De Morgan law implies that $\NAND(x, y) = \neg x \wedge \neg y$, 
therefore \NAND{} formulas are equivalent to ``alternating OR/AND'' formulas (up to negating some inputs), 
that is formulas whose gate with even-depth \footnote{The root has depth 0.} 
are OR gates, and odd-depth gates are AND gates.
This alternative alternating point of view will be useful 
when describing a procedure to compute the value of a formula with alternating computation.

A family of formulas is a sequence $(\phi_n)_{n\in\NN}$ of formulas such that
for every i, $\phi_i$ is a formula over $i$ inputs.
A language $L\subseteq \bit^*$
is computed by\footnote{In what follows, we may use ``$L$ has formulas ...'' for ``$L$ is computed by a family of formulas ...''} 
the family of formulas $(\phi_n)_{n\in\NN}$
if for every $n$ and every $x\in\bit^n$, $\phi_n(x) = 1 \Leftrightarrow x\in L$.

However, there exists families formulas of \textit{constant} depth that compute undecidable languages.
In this work, we restrict our attention to uniform formulas, 
that is family of formulas for which the description of the
$n$-input formula can be computed efficiently.
We first recall a usual definition of uniform formulas.
\begin{definition}[``Usual'' polylog-time uniform formulas]
	A family of formulas $(\varphi_n)_{n \ge 1}$ of depth $(c + o(1))\log n$ 
	is \textit{polylog-time uniform} if there exists a machine $A_\varphi$
	such that for every $n$ and every $b\in\bit^{\leq (c + o(1))\log n}$,
	$A_\varphi(b, n)$ outputs a description of the gate at position $b$ in $\varphi_n$ in time $\log^{O(1)} n$.
\end{definition}

Here, the position of a gate is given by a bit string of length at most $(c + o(1))\log n$ 
that describes the path to follow to reach that gate from the root, 
with $0$ meaning ``go to the left child'' and $1$ ``go to the right child''.

However, in out work, we need to use a stronger model of uniform formulas, 
where the descriptor machine $A_\varphi$ has random access to the input $x$ 
at which we wish to evaluate the formula but is still limited to polylogarithmic time.
\begin{definition}[Input-aware polylog-time uniform formulas]\label{def:unif}
	A family of formulas $(\varphi_n)_{n \ge 1}$ of depth $(c + o(1))\log n$ is \textit{polylog-time uniform}
	if there exists a \textbf{RAM} machine $B_\varphi$
	such that for every $n$, every $x\in\bit^n$ and every 
	$b\in\bit^{\leq (c + o(1))\log n}$,
	$B_\varphi(x, b, n)$ outputs a description of the gate at position $b$ in $\varphi_n$ in time $\log^{O(1)} n$.
\end{definition}
Notice that this model supersedes the previous one: 
$A_\varphi$ satisfies the conditions of Definition~\ref{def:unif}, 
but does not use its random access to the input, 
therefore the lower bounds that we give against this model also hold for the usual one.

We will see the need for this stronger definition in Section~\ref{sec:rules}, 
where we need to give $O(\log n)$ bits of information to the descriptor
to adaptively choose a subformula.

Finally, for any $c \ge 0$, we define $\NDL{c}$
to be the set of languages that have uniform 
input-aware polylog-time NAND formulas of depth$(c+o(1)) \log n$.

Notice that uniform log-depth NAND formulas
can be efficiently simulated by RAM machines, 
due to the negligible runtime overhead of the descriptor.
Indeed, for every $c > 1$, we have $\NDL{c} \subseteq TS[n^c]$.

\subsection{Quantified/Alternating computation}

We now introduce alternating complexity classes, a fine-grained version of the polynomial hierarchy.

Given a complexity class $\Cc$ and constants $k, (a_i)_{i\leq k}$, 
we consider \textit{alternating} complexity
class $(Q_1~n^{a_1})\ldots(Q_k~n^{a_k})~\Cc$, 
where $Q_i$ is one of $\forall$ or $\exists$
A computation in a alternating complexity class is defined as follows:
\begin{definition}[Alternating computation]
	A language $L$ is in $(Q_1~n^{a_1})\ldots(Q_k~n^{a_k})~\Cc$ 
	is defined by $k$ RAM machines $A_1, \ldots, A_k$ and a $\Cc$ machine $M$
	such that, given an input $x$ of length $n$:
	for every $i \le k$, $A_{i}$ takes as input $x_i$ along with $y_i$, 
	a string of $n^{a_{i} + o(1)}$ nondeterministic bits,
	runs in time $n^{a_{i} + o(1)}$ and outputs a string $x_{i}$ of length at most $n^{a_{i} + o(1)}$, 
	with the convention that $x_0 := x$.
	The output of the last stage, $x_{k}$, is passed to the machine $M$. 

	The computation accepts $x$ \textit{starting from stage $i$} if either:
	\begin{itemize}
		\item $i = k+1$ and $M$ accepts $x_{k}$, 
		\item $Q_i$ is $\exists$, and there exists a $y_i$ such that the computation
		accepts $x_{i}$ starting from stage $i+1$, 
		\item $Q_i$ is $\forall$, and for every $y_i$, the computation
		accepts $x_{i}$ starting from stage $i+1$. 
	\end{itemize}

	An input $x$ is in $L$ if the computation $x$ accepts starting from stage 1.
\end{definition}

\subsection{Overview of the methodology}\label{sec:overview}

The general methodology used in this paper to obtain lower bounds
extends the one used in~\cite{williams2006inductive, mudigonda2020time}
to the specific case of uniform \textsf{NAND} formulas of logarithmic depth.

To show that \SAT{} does not have formulas of depth less than $c\log n$, 
we reason by contradiction.
First, we assume that \SAT{} has such formulas, and the strong \NP-completeness of \SAT{}
allows us to deduce that the same holds for all of $\NTIME[n]$, 
\begin{lemma}\label{lemma:usual_sd}
	If \SAT{} is in $\NDL{c}$ for some $c>0$,
	then $\NTIME[n] \cup \coNTIME[n] \subseteq \NDL{c}$.
\end{lemma}

We also show how to speed up (or rather, reduce the depth of) computations by adding quantifiers, 
using various ``Speedup rules''. 
For example, for NAND formulas (which are the same as alternating AND/OR formulas), we have the following:
\begin{lemma}[Speedup rule]
	For every $c > 0, x \leq c$, we have:
	\[\NDL{c} \subseteq (\exists n^{x/2}) (\forall n^1) \NDL{(c-x)}\]
\end{lemma}


The above lemma can be combined with exhaustive search and Lemma~\ref{lemma:usual_sd} 
to yield an improved version of Lemma~\ref{lemma:usual_sd} for quantified classes, 
which we call the ``Slowdown rule''.
\begin{lemma}[Slowdown rule]
	Assume that \SAT{} is in $\NDL{c}$ for some $c>0$.
	Then, for every $a,b,d > 0$, we have:
	\[\ldots (Q~n^{a}) (\neg Q~n^b) \NDL{d}
	\subseteq \ldots (Q~n^{a}) \NDL{c\cdot\max(d/2, a, b)}.\]
\end{lemma}

To prove our results, 
we combine multiple applications of these rules with suitable parameters,
with the goal of contradicting a hierarchy theorem.
In this work, we aim to reach a contradiction of the following hierarchy
theorem for NAND formulas, that we obtain from the assumption that
$\NTIME[n] \subseteq \NDL[c]$ for some $c$, 
as a corollary of the Nondeterministic Time Hierarchy Theorem:
\begin{theorem}\label{thm:nandh}
	If there exists $c > 0$ such that $\NTIME[n] \subseteq \NDL{c}$, 
	then for all $b > a > 0$, we have
	\[\NDL{a} \subsetneq \NDL{b}.\] 
\end{theorem}

For example, we can prove that SAT does not have NAND formulas 
of depth less than $2.8284 \log n$ with the following proof.
Assume that $\SAT\in \textsf{NAND-depth}[c \log n]$ for some $c > 0$.
Then, we have
\begin{flalign*}
	\NDL{4}
		&\subseteq (\exists n)(\forall n)\NDL{2}& \text{Speedup rule with param. } x = 2\\
		&\subseteq (\exists n)\NDL{c}	& \text{Slowdown rule}\\
		&\subseteq \NDL{c^2/2}	& \text{Slowdown rule}
\end{flalign*}
For $c^2/2  < 4 \Leftrightarrow c < 2\sqrt{2} \simeq 2.8284\ldots$,
this contradicts Theorem~\ref{thm:nandh}.

In practice, we use (integer) linear programming to find the
best sequence of rules to apply, along with the optimal parameters.
See Appendix~\ref{app:ilp} for a detailed presentation of the models that we use.

\paragraph{Optimality of the result}
In the case of \SAT{} vs small-space algorithms, 
Buss and Williams~\cite{buss2015limits} proved that the result $\SAT\notin \TS{n^c}$
for $c < 2\cos(\pi/7)$ is optimal for their framework. 
It was later extended and simplified by Mudigonda and Williams~\cite{mudigonda2020time},
who showed upper bounds on .
It appears that their generalization covers our framework,
therefore we cannot prove $\SAT\notin\NDL{c}$ for $c \ge 4\cos(\pi/7)$
without finding a new (and improved) way to speed up \ND{} computation using alternations.

%%%%%%%%%%%%%%%%%%%%%%
%%%%%%%%%%%%%%%%%%%%%%
\section{Proof of the main result}

\subsection{Speedup and Slowdowns}\label{sec:rules}
We start by discussing the Speedup rule for Alternating formulas.

\begin{lemma}[Speedup rule]\label{lemma:speedup}
	For every $d > 0$ and $0 \ge x \ge d$, we have
	\begin{flalign*}
	\NDL{d} &\subseteq (\exists n^{x/2}) (\forall n) \NDL{d-x}\text{ and}\\
	\NDL{d} &\subseteq (\forall n^{x/2}) (\exists n) \NDL{d-x}
	\end{flalign*}
\end{lemma}

The intuitive idea behind this lemma is that the only
information needed to know that an OR gate evaluates
to 1 is that one of its children evaluates to 1. 
We apply this idea to every OR gate in the first $x \log n$
levels of the formula,
and then use a universal quantifier to handle the remaining AND gates.

The role of the existential quantifier is to guess for every OR gate in the first $x\log n$ levels,
which of its children evaluates to 1, and we only evaluate the corresponding subformula.
Moreover, making this guess reduces the number of OR gates that we will evaluate
(since the formula alternate between levels of OR and AND gates, this number doubles every two levels,
instead of doubling at every level), 
therefore we only need to guess $O(n^{x/2})$ bits. 

We can then eliminate OR gates and replace them with their selected child.
We are left with a formula starting with $(x/2)\log n$ levels of AND gates, 
followed by smaller OR/AND formulas of depth $(d-x + o(1)) \log n$.
The main formula evaluates to 1 if and only if for every of these $O(n^{x/2})$
smaller formulas evaluates to true: 
we can check this using a universal quantifier, 
which requires $O(\frac{x}{2}\log n)$ bits, and then running an $\NDL{d-x}$ computation.
Note that while the second quantifier only requires $O(\frac{x}{2}\log n)$ nondeterministic bits,
it still computes and outputs $n^{1+o(1)}$ bits, hence the $n^1$ exponent. 

Finally, notice that the definition of alternating computation
requires that there exists a \textit{single} machine $M$ that runs the
final stage of computation \textit{for every} possible choice of nondeterministic bits.
This is where we use the strong notion of uniformity (from Definition~\ref{def:unif}), where the descriptor machine
has random access to the input: giving $M$ access to (a few) bits of the input
allows us to adapt its behavior depending on the nondeterministic bits. 

\begin{proof}[Proof of Lemma~\ref{lemma:speedup}]
	We prove the result for $\exists\forall$ computations, 
	the result for $\forall\exists$ computations follows from the fact
	that $\NDL{d}$ is closed under taking the complement.
	
	Let $L\in\NDL{d}$, let $M : (x,i,n) \mapsto g$ be 
	a descriptor machine for $L$, and let $x$ be an input of length $n$.
	
	We now construct a machine $M'$ that describes $\NDL{d-x}$-formulas,
	such that an $\exists\forall$ computation with $M'$ as its last stage accepts $L$.
	On an input $x$

	We describe how to build bit strings $y$ of length $n^{x/2}$ and
	\todoh{finish details of proof}
\end{proof}

Lemma~\ref{lemma:speedup} yields the following corollary,
which gives us a simulation of formulas by Turing machines that is more efficient
than the naive evaluation.
\begin{corollary}\label{cor:speedup}
	For every $d > 0$, we have
	\[\NDL{d} \subseteq \NTIME[n^{d/2}] \cap \coNTIME[n^{d/2}].\]
\end{corollary}
\begin{proof}
	Applying Lemma~\ref{lemma:speedup} with $x = d$, we obtain
	$\NDL{d} \subseteq (\exists n^{d/2}) (\forall n) \NDL{0}$.
	However, looking closely at the proof of Lemma~\ref{lemma:speedup}
	reveals that the $\forall$ quantifier here only uses $O(\frac{d}{2}\log n)$
	nondeterministic bits.
	Therefore, we can use exhaustive search over the $n^{d/2}$ bit strings of length $\frac{d}{2}\log n$
	to remove that quantifier.
	This leaves us we a deterministic $n^{o(1)}$ computation, and we get that
	$\NDL{d} \subseteq (\exists n^{d/2}) \TIME[n^{d/2}] = \NTIME[n^{d/2}]$.

	The inclusion in $\coNTIME[n^{d/2}]$ follows from the stability of $\NDL{d}$
	under complement.
\end{proof}

We now discuss how to use this corollary to
efficiently remove quantifiers from an alternating \ND{} computation.
The first step is to show that the assumption $\SAT\in\NDL{c}$
propagates efficiently to all of \NP{}.
This result is known as the Slowdown rule: it allows us to remove a quantifier,
at the cost of an increased depth of the verifier.
\begin{lemma}[Slowdown rule]\label{lemma:slowdown}
	If $\SAT{}\in \NDL{c}$, then for every $d \ge 1$, $\NTIME[n^d] \cup \coNTIME[n^d] \subseteq \NDL{c\cdot d}$.
\end{lemma}

A key idea for the proof of Lemma~\ref{lemma:slowdown} 
is the fact that any language in $\NTIME[n]$ can
be reduced to $\SAT{}$ via uniform poly-log time first-order projections, 
with instances of size $n$ reduced to formulas of size $n^{1+o(1)}$.
Intuitively, a first-order projection is a depth 0 circuit, 
i.e. each output bit is either a constant (0 or 1),
an input $x_i$, or its negation $\neg x_i$.
For a more in-depth exposition of first-order projections,
see for example~\cite[End of Sec.~3]{allender1997first}.
\begin{proof}[Proof of Lemma~\ref{lemma:slowdown}]
	We first prove that $\SAT{}\in \NDL{c}$ implies that $\NTIME[n] \subseteq \NDL{c}$
	and use a padding argument to show that 
	\[\forall d\ge 1, \NTIME[n^d] \subseteq \NDL{c\cdot d}.\]
	The result follows since $\NDL{c\cdot d}$ is closed under complement.

	Let $L\in\NTIME[n]$. 
	As stated above, $L$ is Karp-reducible to \SAT{} via
	uniform poly-log time first-order projections, 
	with instances of size $n$ reduced to formulas of size $n^{1+o(1)}$.
	Now, notice that the composition of a NAND formula of depth $d$ with a first-order projection
	is also a NAND formula of depth $d$: it only consists in relabeling leaves of the formula.
	Poly-log time uniformity is also preserved. 
	Therefore, we can use $(c+o(1)) \log n$-depth NAND formulas for \SAT{} on $n^{1+o(1)}$
	inputs to decide $L$ by composing them with the first-order projection; 
	the resulting formula has size $(c+o(1)) \log n^{1+o(1)} = (c+o(1)) \log n$,
	therefore $L\in\NDL{c}$.

	We now use a padding argument to lift this result to $\NTIME[n^d]$.
	Let $d>1$, let $L\in \NTIME[n^d]$ and let $M$ be a nondeterministic machine that decides $L$
	in time $n^{d+o(1)}$.
	Consider the padded language $L' = \{x1^{|x|^d - |x|} \mid x\in L\}$.
	Given an input $y$ of length $m$, we can decide whether $y\in L'$ in time 
	$m^{1+o(1)}$ as follows:
	\begin{itemize}
		\item if $y$ is not of the form $x1^{|x|^d - |x|}$ for some $x$, reject $y$.
		This can be done in time $O(m)$.
		\item otherwise, $y = x1^{|x|^d - |x|}$. Run $M$ on $x$, and accept $y$ 
		if and only it accepts $x$. This takes time $|x|^{d+o(1)} = m^{1+o(1)}$.
	\end{itemize}
	This implies $L'$ is in $\NTIME[n]$, which in turn implies that $L'\in\NDL{c}$.
	We can then use the formula descriptor $A$ for $L'$ 
	to describe formulas of depth $(c + o(1))\log \left(n^{d+o(1)}\right) = (c\cdot d + o(1))\log n$ for $L$.
	On input $x,b,n$, the descriptor machine $B$ for $L$  
	calls $A$ on $x, b, n^d$.
	If $A$ outputs a gate type or a constant, $B$ outputs the same information.
	If $A$ outputs a literal $x_i^*$, $B$ outputs $x_i^*$ if $i \le n$, and 1 otherwise.
\end{proof}

We need to be careful when applying the slowdown lemma to alternating classes:
the $i+1$-th computation stage takes as input a string of length $m = n^{a_i}$, not $n$.
Since the previous result only holds for $\NTIME[n^d], d \ge 1$, 
we need to take into account the edge case where the running time of
the $i+1$-th stage is sublinear in the size of its input.

We will now show that applying the Speedup rule once and
using exhaustive search to remove the logarithmic-size quantifier
yields an improved Slowdown rule for quantified classes.
We have the following:
\begin{corollary}[Extended Slowdown rule]\label{cor:extended_slowdown}
	If $\SAT{}\in \NDL{c}$, then for every $d > 1$, we have:
	\[\ldots (Q~n^{a}) (\neg Q~n^b) \NDL{d}
	\subseteq \ldots (Q~n^{a}) \NDL{c\cdot\max(d/2, a, b)}.\]
\end{corollary}
\begin{proof}
	Without loss of generality, assume that $Q = \forall$, and therefore $\neg Q = \exists$.
	In that case, for all $q, r > 0$, we have $(\exists n^q) \NTIME[n^r] \subseteq \NTIME[n^{\max(q,r)}] $.
	Applying Corollary~\ref{cor:speedup} on the $\NDL{d}$ part and then using the 
	above observation leads to the following:
	\begin{flalign*}
	\ldots (\forall n^{a}) (\exists n^b) \NDL{d} 
		&\subseteq \ldots (\forall n^{a}) (\exists n^b) \NTIME[n^{d/2}]\\	
		&\subseteq \ldots (\forall n^{a}) \NTIME[n^{\max(d/2, d)}]	
	\end{flalign*}

	Now, consider the $\NTIME[n^{\max(d/2, b)}]$ language $L$
	that correspond to the last stage of the alternating 
	computation, and its corresponding nondeterministic machine $M$. 
	It takes as input a string of length $m = n^{a}$, and its runtime, as a function of $m$, 
	is $m^{\max(d/2, b)/a}$.
	We apply the Slowdown rule (Lemma~\ref{lemma:slowdown}) to $M$. There are two possible cases:
	\begin{itemize}
		\item if $\max(d/2, b)/a < 1$, then we have 
		$L \in \NTIME[m^{\max(d/2, b)/a}] \subseteq \NTIME[m]
		\subseteq \ND[c \log m] \subseteq \NDL{c \cdot a}$.
		Since  $\max(d/2, b) < a$, we have $a = \max(d/2, a, b)$ and therefore 
		$\NDL{c \cdot a} = \NDL{c\cdot\max(d/2, a, b)}$.
		\item otherwise, $\max(d/2, b) \ge a$, and applying the slowdown rule yields
		$L \in \NTIME[m^{\max(d/2, b)/a}] =  \NTIME[n^{\max(d/2, b)}] \subseteq \NDL{c \cdot \max(d/2, a, b)}$.
	\end{itemize}
\end{proof}

%%%%%%%%%%%%%%%%%%%%%%
\subsection{Conditional Depth hierarchy theorem for \textsf{NAND} formulas}

We now prove the Conditional Depth hierarchy theorem, 
stated without proof in Section~\ref{sec:overview}.
\begin{theorem}
	If there exists $c > 0$ such that $\SAT \subseteq \NDL{c}$, 
	then for all $b > a > 0$, we have
	\[\NDL{a} \subsetneq \NDL{b}.\] 
\end{theorem}
\begin{proof}
	
\end{proof}

Note that this theorem relies heavily on the assumption that $\SAT\in\NDL{c}$,
hence this theorem is \textit{conditional}.
It is an open problem to give an unconditional depth hierarchy theorem that 
separates $\NDL{a}$ from $\NDL{b}$ for any $a < b$.

%%%%%%%%%%%%%%%%%%%%%%
\subsection{Structure of optimal proofs}
In this section, we describe
the process that helps us build a proof, formed of successive applications
of the speedup and slowdown rules, 
that reaches a contradiction from the assumption
$\SAT\in\NDL{c}$ for some $c < 4\cos(\pi /7)$.

\begin{observation}
	Consider a quantifier trading proof starting from $\ldots (Q_k~n^{a_k})\NDL{d}$
	and that never goes below $k$ quantifiers.
	Then, for the resulting class $\ldots (Q_k~n^{a_k})\NDL{d'}$, we have $d' \ge c\cdot a_k$.
\end{observation}
This is due to the presence of $a$ in the $\max$ in the Slowdown 
rule (Corollary~\ref{cor:extended_slowdown}).
In what follows, we aim to reach $d' = c\cdot a_k$ 
and want to apply this process inductively,
therefore we will start from depth $d = \frac{c^2}{2}a_{k+1}\log n$
(which result from a slowdown on depth $ca_{k+1}\log n$).
Not all values can be reduced to $c\cdot a_k$: 
a study of necessary and sufficient conditions
will give us a lower and an upper bound on $a_{k+1}$ in terms of $c$ and $a_k$,
from which we will derive an explicit construction of an optimal choice of values for $(a_k)_k$.

%%%%%%%
\paragraph{Reducing the depth to $c a_k \log n$}
The critical part is to manage to reduce the depth of the verifier
from $\frac{c^2}{2}\cdot a_{k+1} \log n$ to $ca_k \log n$,
while preserving the last quantifier exponent to $a_k$.
One way to do this is by applying a sequence of speedup-slowdown alternation.
One application of speedup then slowdown proves the following:
\begin{flalign*}
	\ldots (Q_k~n^{a_k})\NDL{d} 
		&\subseteq \ldots \left(Q_k~n^{\max(a_k, x)}\right) (Q_{k+1}~n) 
			\NDL{(d-2x)} \text{ Speedup w/param.} x\\ 
		&\subseteq \ldots \left(Q_k~n^{\max(a_k, x)}\right) 
			\NDL{c\cdot\max(a_k, x, (d-2x)/2, 1)} \text{ Slowdown} 
\end{flalign*}
Since we do not want to increase the exponent of the $Q_k$ quantifier,
we get the following constraint:
\begin{equation}
	x \le a_k \label{eq:cstr1}
\end{equation}
Moreover, assuming that $a_k$ is at least $1$ allows us to reformulate
the last class as 
\[\ldots \left(Q_k~n^{a_k}\right) \NDL{c\cdot\max(a_k, (d-2x)/2)}.\]

From here, we can get the following:
\begin{lemma}[``Squiggle rule'']\label{lemma:squiggle}
	Let $a,d$ be positive real numbers such that $ac < d < \frac{2ca}{c-2}$.
	Then, the proof $(1_{a}0)^t, t \le \left\lceil\frac{d}{(4-c)a}\right\rceil + 1$
	proves that
	\[\ldots (Q~n^a)\NDL{d} \subseteq \ldots (Q~n^a)\NDL{ca}.\]
\end{lemma}
\begin{proof}
	If $d/2 - a \le a \Leftrightarrow d \le 4a$,
	then we get the desired result.
	Setting $x = d/2 - a_k$ would immediately yield the desired depth.

	We now assume that $d > 4a$, and show that we can get to $d \le 4a$ in a finite number of steps.
	When $d \ge 4a$, applying the speedup rule with parameter $x = a$
	followed by the slowdown rule transforms
	$d$ is into $c\cdot(d/2 - a)$.

	This is an improvement whenever $d$ is such that:
	\begin{equation}
		d < \frac{2ca_k}{c-2} \label{eq:cstr3}
	\end{equation}
	Moreover, as long as $c < 4$, we have $\frac{2c}{c-2} > 4$, hence 
	there is a gap between $4a$ and $\frac{2ca}{c-2}$ where we can get a non-trivial improvement.

	The improvement is $d(1 - c/2) + ca > (4-c)a$, since we are in the case $d > 4a$.
	Therefore, as the improvement is lower bounded by a constant,
	iterating this process reaches a value below $4a$ 
	starting from $d$ in $\left\lceil\frac{d}{(4-c)a}\right\rceil$ steps,
	after which one more step with param $x = d/2 - a$
	yields the desired $ca \log n$ depth.
\end{proof}

%%%%%%%
\paragraph{Combining all the pieces}
As shown above, we need that $d < \frac{2ca_k}{c-2}$ 
to be able to reach $ca_k$ starting from $d$.
Since we want to apply this argument inductively, we will have 
$d = \frac{c^2}{2}a_{k+1}$.
Combining the above yields the following:
\begin{equation}
	a_{k+1} < \frac{4a_k}{c(c-2)} \label{eq:cstr_ak}
\end{equation}
Notice that $\frac{4}{c(c-2)} < 1$ whenever $c > 2\phi = 1 + \sqrt{5}$, 
where $\phi$ denotes the golden ratio.

Eq.~\ref{eq:cstr_ak} leads us to the following definition.
\begin{definition}
	Let $c > 2\phi$.
	A sequence of real number $(a_k)_{k=0,\ldots,t}$ is \textit{well-behaved for $c$}\todo{find a name for that}
	if the following holds:
	\begin{enumerate}
		\item for every $k=0,\ldots,t, a_k \ge 1$,
		\item $a_t = 1$,
		\item for every $k=0,\ldots,t, c a_{k+1}/2  \ge a_k$, and
		\item\label{constr:decr} for every $k=0,\ldots,t-1, a_{k+1} < \frac{4a_k}{c(c-2)}$.
	\end{enumerate}
\end{definition}
Notice that constraint~\ref{constr:decr} above implies that $a_{k+1} < a_k$,
as $c > 2\phi$ implies that $\frac{4}{c(c-2)} < 1$.

\begin{lemma}
	Let $c > 2\phi$, let $(a_k)_{k=0,\ldots,t}$ be a sequence well-behaved for $c$,
	and let $d_0 = 2\left(1+\sum_{k=0}^t a_k\right)$.
	Then there exists a alternation trading proof that shows
	\[\NDL{d_0} \subseteq \NDL{\frac{c^2}{2}a_0}.\]
\end{lemma}
\begin{proof}
	The proof starts by applying $t$ speedups, with respective parameters $a_1,\ldots, a_t$.
	We obtain the following:
	\begin{flalign*}
		\NDL{d_0}
			&\subseteq (\exists n^{a_0}) (\forall n) \NDL{d_0 - 2_a0}\\
			&\subseteq (\exists n^{a_0}) (\forall n^{a_1}) (\exists n) \NDL{d_0 - 2a_0 - 2a_1}\\
			&\subseteq \vdots\\
			&\subseteq (\exists n^{a_0}) \ldots (Q_t n^{a_t}) (Q_{t+1} n) \NDL{d_0 - 2\sum_{k=0}^t a_k}\\
			&= (\exists n^{a_0}) \ldots (Q_t n^{a_t}) (Q_{t+1} n) \NDL{2}
	\end{flalign*}
	Then, applying Slowdown rule yields the class
	\[(\exists n^{a_0}) \ldots (Q_t n^{a_t}) \NDL{c\cdot a_t}.\]
	We will now repeat $t$ successive applications of the Slowdown rule and Squiggle rule.
	Informally, it will give us the following inclusions:
	\begin{flalign*}
			(\exists n^{a_0}) \ldots (Q_t n^{a_t}) \NDL{c\cdot a_t}
			&\subseteq (\exists n^{a_0}) \ldots (Q_{t-1} n^{a_{t-1}}) \NDL{\frac{c^2}{2}\cdot a_t} \text{ (Slowdown)} \\
			&\subseteq (\exists n^{a_0}) \ldots (Q_{t-1} n^{a_{t-1}}) \NDL{c\cdot a_{t-1}} \text{ (Squiggle)} \\
			&\vdots \\
			&\subseteq (\exists n^{a_0}) \NDL{c\cdot a_0} \\
			&\subseteq \NDL{\frac{c^2}{2}a_0} \text{ (Slowdown)}
	\end{flalign*}
	
	This can be shown formally by induction, mainly using the fact that $(a_k)_k$
	is well-behaved for $c$:
	during the $i$-th step ($i = 0,\ldots, t-1$), we have the following properties:
	\begin{enumerate}
		\item\label{inv1} before applying the Slowdown rule,
		the depth of the verifier is $c\cdot a_{t-i}\log n$,
		\item\label{inv2} after applying the Slowdown rule,
		the depth is $\frac{c^2}{2}a_{t-i}\log n$,
		\item\label{inv3} after applying the Squiggle rule,
		the depth is $c\cdot a_{t-i-1}\log n$.
	\end{enumerate}
	A step $i = t$, we only apply the Slowdown rule
	(we cannot apply the Squiggle rule: there is no leading quantifier).

	Property~\ref{inv1} is true for $i = 1$, and then holds by induction when Property~\ref{inv3}
	holds for $i-1$.
	Property~\ref{inv2} hold when Property~\ref{inv1} holds, 
	by definition of the Slowdown rule (Corollary~\ref{cor:extended_slowdown}),
	and using the fact that $a_i \ge 1$ and $ca_i > a_{i-1}$.
	Finally, Property~\ref{inv3} hold when Property~\ref{inv2} holds:
	we can apply the Squiggle rule (Lemma~\ref{lemma:squiggle}) since $a$ is well-behaved for $c$,
	which ensures that $i=0,\ldots,t-1, a_{i+1} < \frac{4a_i}{c(c-2)}$.
\end{proof}

The last step is to, given a $c < 4\cos(\pi/7)$, construct a sequence well-behaved for $c$
such that ${d_0} > \frac{c^2}{2}a_0$, which yields the desired contradiction.
The following lemma constructs such a sequence explicitly.
\begin{lemma}\label{lemma:construction}
	Let $c \in ]2\phi, 4\cos(\pi/7)[$, and let $\tau = \frac{c(c-2)}{4}$.
	Then there exists an integer $t$ such that the sequence
	$(r\cdot\tau^{t-i} - \eps)_{k=0,\ldots, t}$ is well-behaved for $c$, where $r = 1 + \eps$,
	and $d_0  = 2\left(1+\sum_{k=0}^t a_k\right)$ is greater than $\frac{c^2}{2}a_0$.
\end{lemma}
\begin{proof}
	We first check that $(a_k)_k$ is well behaved for $c$.
	Notice that since $r = 1 + \eps$, we have $a_t = 1$,
	and since $\tau > 1$, we have $a_k \ge 1$ for every $k = 1,\ldots, t$.
	Now, for every $k < t$, we have $a_k + \eps = \tau(a_{k+1} + \eps)$,
	which, combined with $\tau > 1$ implies that
	\[a_{k+1} = a_k/\tau +\eps(\tau^{-1} - 1) < \frac{a_k}{\tau} = \frac{4a_k}{c(c-2)}.\]
	The constraint $ca_{k+1} \ge a_k$ can we rewritten as
	\[a_{k+1} (\frac{c}{2} - \tau) \ge \eps (\tau -1).\]
	Since $\eps = 1/(t+1)$, the RHS of this inequality goes to $0$ as $t$ goes to infinity,
	and $(\frac{c}{2} - \tau) > 0$ when $c$ is greater than $3$ (which is the case here since $c > 2\phi$).
	Therefore, there exists a $t$ such that the inequality holds for every $k$, 
	and the sequence $(a_k)_{k=1,\ldots,t}$ is well-behaved for $c$.

	We now show that there exists a $t$ such that $d_0 > \frac{c^2}{2}a_0$.
	We have:
	\begin{flalign*}
	d_0 
		&= 2\left(1 - (t+1)\eps + \sum_{k=0}^t r\tau^k\right)\\
		&= 2\left(1 - 1 + r\sum_{k=0}^t \tau^k\right)\\
		&= 2r\frac{\tau^{t+1}-1}{\tau-1}
	\end{flalign*}
	Therefore, we have
	\begin{flalign*}
		d_0 > \frac{c^2}{2}a_0 
		&\Leftrightarrow 2r\frac{\tau^{t+1} -1}{\tau-1} > \frac{c^2a_0}{2} \\
		&\Leftrightarrow r\frac{\tau\tau^t  - 1}{\tau - 1} > \frac{c^2(r\tau^t-\eps)}{4} \\
		&\Leftrightarrow r\tau^t\left[\frac{c^2}{4} - \frac{\tau}{\tau - 1}\right] < \frac{-1}{\tau - 1} + \eps\frac{c^2}{4}\\
		&\Leftrightarrow \frac{c^2}{4}\tau - \frac{c^2}{4} - \tau < (r\tau^t)^{-1}\left[- 1 + \frac{c^2(\tau - 1)}{4(t+1)}\right]\\
		&\Leftrightarrow \frac{c^2}{4}\tau - \frac{c^2}{4} - \tau < \frac{-1}{(1+\eps)\tau^t} + \frac{c^2(\tau - 1)}{4(t+1)(1+\eps)\tau^t}
	\end{flalign*}
	We can rewrite the LHS of the last inequality as follows:
	\begin{flalign*}
		\frac{c^2}{4}\tau - \frac{c^2}{4} - \tau 
			&= \frac{c^3(c-2)}{16} - \frac{c^2}{4} - \frac{c(c-2)}{4}\\	
			&= \frac{1}{4}\left(\frac{c^4}{4} - \frac{c^3}{2} - 2c^2 + 2c\right)
	\end{flalign*}
	Moreover, the RHS of that same last inequality is negative
	and goes to $0$ as $t$ goes to infinity, 
	while the LHS does not depend on $t$.
	Therefore, for every $\delta < 0$, there exists a $t$ such that the last inequality is satisfied
	whenever we have:
	\[c^3 - 2c^2 - 8c + 8 < \delta.\]
	The polynomial $P = X^3 - 2X^2 - 8X + 8$ is negative whenever $X$ is between
	its 2nd largest $r_2 \simeq 0.89$ and largest root $r_1 = 4\cos(\pi/7)$.
	
	Now, fix some $c \in ]r_2, r_1[$, and let $\delta = P(c) / 2 > P(c)$.
	From the above, there exists a $t$ such that $c^3 - 2c^2 - 8c + 8 < \delta$
	implies $d_0 > \frac{c^2}{2}a_0$, and therefore
	the proof for $(a_k)_k$ leads to a contradiction
	assuming $\SAT\in\NDL{c}$.
\end{proof}

Lemma~\ref{lemma:construction} proves Theorem~\ref{thm:main} for $c \in]2\phi, 4\cos(\pi/7)[$;
to extend this result to $c \le 2\phi$, it suffices to notice that in that case,
we have $\NDL{c} \subseteq \NDL{(2\phi + \eps)}$, hence it reduces to the former case.

%%%%%%%%%%%%%%%%%%%%%%
%%%%%%%%%%%%%%%%%%%%%%
\section{Can we turn depth lower bounds into (good) size lower bounds?}
Can we prove an improved Brent/Spira-type theorem that turns our lower bounds into super-cubic size lower bounds?
Useful starting point: Size-Depth Tradeoffs for Boolean Formulae (Bonet, Buss).


%%%%%%%%%%%%%%%%%%%%%%
%%%%%%%%%%%%%%%%%%%%%%
\section{Conclusion and future work}
The lower bounds given in this work are against a very restricted
type of formulas (NAND formulas, or equivalently alternating OR/AND formulas).
An open problem is to lift these lower bounds to other computational bases,
for example by allowing XOR and equality gates,
using ``even'' or ``odd-parity'' quantifiers, 
and assuming that \ParSAT{} has low-depth formulas.
In that case, we can get a lower bound for \SAT{} \textit{or} \ParSAT{}.
We can go further, and get lower bounds for \SharpSAT{},
since \ParSAT{} is the least significant bit of \SharpSAT{}.
The first idea to solve \SAT{} (that is, to remove existential quantifiers) 
with \SharpSAT{} would be to take the OR of all $n+1$
bits of \SharpSAT{}, but this add depth $\log n$ to the formula.
However, \textsf{Tautology} can be solved
with the most significant bit of \SharpSAT{}:
therefore we may be able to remove existential quantifiers 
by looking at a single bit of \SharpSAT{}
for a formula that corresponds to a negation of the verifier.

We may also be able to obtain results against less-restricted formulas,
for example by allowing two types of gates within a layer,
while keeping an alternating structure.
For example: odd-numbered layers are ANDs and ORs mixed,
even-numbered layers are XORs and EQs mixed.

We can also relax the structure of the formulas: 
instead of alternating layers, we can only ask
that at least one layer in every group of $k$ layers
is an OR layer, and at least one other is an AND layer.
We get weaker speedups and slowdowns
(namely, the $1/2$ in the exponent becomes $\alpha_k = (k-1)/k$),
but against more expressive formulas. 


%%%%%%%%%%%%%%%%%%%%%%
%%%%%%%%%%%%%%%%%%%%%%
%%%%%%%%%%%%%%%%%%%%%%
\appendix
%%%%%%%%%%%%%%%%%%%%%%
%%%%%%%%%%%%%%%%%%%%%%
\section{Program to find optimal parameters}\label{app:ilp}

\subsection{Overview of the approach}
Improves over Williams (cite): no need to enumerate proof annotations.

Therein, Williams enumerates all possible proof skeletons of length $k$,
and tries to find the best set of parameters for each of them, using a linear program.
In this work, we propose a different approach, that proves to be more efficient in practice:
we use an Integer Linear program whose solution is
an optimal proof of length $k$
along with its parameters.

More precisely, we give an ILP $P_{c,k}$ which is feasible if and only if
there exists a proof annotation $p$ of length $k$,\footnote{Our program takes as input a value $k'$ and searches over proofs with $k'$ application of the speedup rule, which implies $k'+1$ slowdowns since every quantifier must be removed to reach a contradiction. Therefore, we have $k = 2k'+1$.} along with parameters $x, r_0$
such that $p$ proves a contradiction under the assumption that $\SAT\in\NDL{c}$
Moreover, its solution encodes $p,q,r_0$.
By using binary search over $c$, we can quickly get an approximation 
of $c_k^*$, the largest $c$ such that a proof of length $k$ can reach a contradiction assuming that $\SAT\in\NDL{c}$.
Repeating the above steps for increasing values of $k$ shows that the sequence $(c_k^*)_{k}$
seems to converge to some value $c_\infty^* > 3.5$ (see Table~\ref{table:ilp_res}).

\begin{table}[htbp]
	\begin{center}
	\begin{tabular}{ |c|l|l| } 
	\hline
	$k$ & $c_k^*$ & Skeleton of $p_k^*$ \\ 
	\hline
	3 & 2.82812 & \texttt{100} \\
	7 & 3.20020 & \texttt{1100100} \\
	11 & 3.32617 & \texttt{11100100100} \\
	15 & 3.39844 & \texttt{111001010010100} \\
	19 & 3.44629 & \texttt{1111001001010010100} \\
	23 & 3.47559 & \texttt{11110010100101001010100} \\
	27 & 3.49805 & \texttt{111110010010100101001010100} \\
	\hline
	\end{tabular}
	\end{center}
	\caption{Approximate value of $c_k^*$ and skeleton of $p_k^*$ for some values of $k$.}\label{table:ilp_res}
\end{table}

\subsection{Details of the model}
Given an integer $k$ and a real $c$, we define the following interger linear program $P_{c,k}$ as follows:
\begin{equation}
	todo
\end{equation}
\todoh{describe model}
\todoh{link to code?}


%%%%%%%%%%%%%%%%%%%%%%%%%
%%%%%%%%%%%%%%%%%%%%%%%%%
\bibliographystyle{plain}
\bibliography{biblio.bib}

\end{document}
