\documentclass[a4paper, 11pt]{article}
% \usepackage{ae,lmodern}
\usepackage[utf8]{inputenc}
\usepackage[T1]{fontenc}
\usepackage[USenglish]{babel}
\usepackage[margin=3cm]{geometry}
\usepackage{amsfonts}
\usepackage{amsmath}
\usepackage{amsthm}
\usepackage{amssymb}
\usepackage{todonotes}
\usepackage{enumitem}
\usepackage{algorithm}
\usepackage[noend]{algpseudocode}


\theoremstyle{plain}
\newtheorem{theorem}{Theorem}[section] % Number within sec ?
% \newtheorem{theorem}{Theorem}
\newtheorem{prop}[theorem]{Proposition}
\newtheorem{property}[theorem]{Property}
\newtheorem{lemma}[theorem]{Lemma}
\newtheorem{corollary}[theorem]{Corollary}
\newtheorem{proposition}[theorem]{Proposition}
\newtheorem{exercise}[theorem]{Exercise}
\newtheorem{conjecture}[theorem]{Conjecture}
\newtheorem{observation}[theorem]{Observation}
\theoremstyle{definition}
\newtheorem{definition}[theorem]{Definition}
\newtheorem{example}[theorem]{Example}
\theoremstyle{remark}
\newtheorem{remark}[theorem]{Remark}
% \newtheorem*{remark*}{Remark}
% \theoremstyle{definition}
% \newtheorem{problem}{Problem}
% \newtheorem{openproblem}{Open Problem}

\newcommand{\FF}{\mathbb{F}}%
\newcommand{\NN}{\mathbb{N}}%
\newcommand{\ZZ}{\mathbb{Z}}%
\newcommand{\Sbb}{\mathbb{S}}%
\newcommand{\Cc}{\mathcal{C}}%
\newcommand{\Gg}{\mathcal{G}}%
\newcommand{\Ff}{\mathcal{F}}%
\newcommand{\Ss}{\mathcal{S}}%
\newcommand{\eps}{\varepsilon}%
\newcommand{\epsm}{\eps^{-1}}%
\newcommand{\bit}{\{0,1\}}%
\newcommand{\mbit}{\{-1,1\}}%
\newcommand{\gbit}{\{g^{-1},g\}}%
\DeclareMathOperator*{\poly}{poly}

\newcommand{\Omegat}[1]{\widetilde{\Omega}\left( #1 \right)}%
\newcommand{\Ot}[1]{\widetilde{O}\left( #1 \right)}%

\newcommand{\Poly}{\textsf{P}}%
\newcommand{\NP}{\textsf{NP}}%
\newcommand{\TIME}{\textsf{TIME}}%
\newcommand{\NTIME}{\textsf{NTIME}}%
\newcommand{\coNTIME}{\textsf{coNTIME}}%
\newcommand{\SAT}{\textsf{SAT}}%
\newcommand{\ParSAT}{\ensuremath\bigoplus\textsf{SAT}}%
\newcommand{\SharpSAT}{\#\textsf{SAT}}%
\newcommand{\NAND}{\textsf{NAND}}%
\newcommand{\ND}{\textsf{NDepth}}%
\newcommand{\NDL}[1]{\ND[ #1 \log n]}%
\newcommand{\ssym}{\textsf{SYM$\circ$SYM}}%

\newcommand{\todohere}{\todo[inline]{TODO}}%

%%%%%%%%%%%%%%%%%%%%%%
%%%%%%%%%%%%%%%%%%%%%%
\title{Fast multipoint evaluation for exponential size CNFs and SYM$\circ$SYM circuits}
\author{Gabriel Bathie, Ryan Williams}
\begin{document}
\maketitle

\begin{abstract}
We introduce a new way to encode size-$m$ \ssym{} circuits with possibly negated inputs
(which include CNFs and DNFs)
into polynomials that can be evaluated over all inputs in $\bit^n$ in time $O^*(2^n + m)$.
As a consequence, we obtain a $O^*(2^n)$-time algorithm that computes the
truthe table of CNFs with $2^n$ clauses.
Lower bounds ensue. (do they though?)
\end{abstract}
\todo[inline]{Todo: issue with weights?}

%%%%%%%%%%%%%%%%%%%%%%
%%%%%%%%%%%%%%%%%%%%%%
\section{Introduction}

\subsection{Results overview}

\begin{theorem}
	For all $n, m$, \ssym{} circuits of size $m$ over $n$ variables, with possibly negated inputs, can be evaluated
	on every input in time $O^*(2^n + m)$.
\end{theorem}

This yields a significant improvement over the exhaustive search algorithm, that takes time $O(2^n\cdot m)$.

\begin{corollary}
	CNFs (and DNFs) on $n$ variables and $m$ clauses can be evaluated
	over all inputs in time $O^*(2^n + m)$.
\end{corollary}

%%%%%%%%%%%%%%%%%%%%%%
%%%%%%%%%%%%%%%%%%%%%%
\section{Definitions}

A symmetric function is a boolean function whose output only depends on the number of 
its input that equal to 1.
More formally:
\begin{definition}[Symmetric functions]
	A boolean function $f: \mbit^n \rightarrow \mbit$ is \textit{symmetric} 
	if there exists a function
	$h: \{-n,\ldots, n-1, n\} \rightarrow \mbit$ such that for every $x\in\mbit^n$:
	\[f(x) = h\left(\sum_{i=1}^n x_i\right).\]
	The function $h$ is called the activation function of $f$.
\end{definition}


\begin{definition}[\ssym{} circuits]
	A \ssym{} circuit is a depth 2 circuit where every gate computes a symmetric function.
	Inputs to the circuit can be negated.
\end{definition}
Notice that we can assume w.l.o.g. that the output of the gates of the bottom layer is not negated, since the negation of a symmetric function is also a symmetric function.

For a symmetric function $f$ and its associated function $h$, we define $T(f) = h^{-1}(1)$,
that is, $T(f)$ is the set of integers $t$ such that $f$ is 1 when exactly $t$ of its inputs are equal to 1.

Additionnally, for a gate $f$, we define $S^+(f)$ (resp. $S^-(f)$)
to be the set of variables that are positive (resp. negative) inputs to $f$.
Finally, let $S(f) = S^+(f) \cup S^-(f)$.

\todo[inline]{TODO: Fix. What if a variables goes TWICE into a gate ? Doesn't matter for CNFs, but otherwise this needs to be a multiset, and we would have $X_i^{2a}$}

%%%%%%%%%%%%%%%%%%%%%%
%%%%%%%%%%%%%%%%%%%%%%
\section{Proof of main result}
Given a \ssym{} circuit $C$ and a enumeration $F$ of the gates on its second layer, we show how to construct a polynomial $P_C$ 
whose value at $(g^{x_i})_{i =1,\ldots n}$ is equal to $\sum_{f \in F} f(x)$.
By applying the activation function of the topmost gate of $C$ to the 
value of $P_C$, we can recover the value of $C$.
Here, $g$ is a well chosen element that generates a small multiplicative subgroup of $\FF_p$.

We then show how to evaluate efficiently $P_C$ on all inputs of $\gbit^n$,
from which we can compute the truth table of $C$.

%%%%%%%%%%%%%%%%%%%%%%
\subsection{Encoding \ssym{} as polynomials}

Let $C$ be a \ssym{} circuit, and let $F$ be an enumeration of the symmetric gates on its second layer. We encode it with the following polynomial:

\[P_C(X) = -|F| + 2q^{-1}\sum_{a = 1}^q \sum_{t = -n}^n g^{-at} \sum_{f \in F : t\in T(f)} \left(\prod_{i \in S^+(f)} X_i^a\right) \left(\prod_{i \in S^-(f)} X_i^{-a}\right)\]

We will evaluate $P_C$ over the finite field $\FF_p$, where $p$ is a prime number larger that $m$. 
Moreover, $q$ is a prime number larger that $2n$ that divides $p-1$, 
and $g$ is an element of order $q$ in $\FF_p$. 
See Section~\ref{sec:primes} for a discussion of the existence of suitable values for $p$ and $q$.

\begin{lemma}\label{lemma:symgate}
	For any symmetric function $f$ and for any $x\in\mbit^n$, we have:
	\[f(x) = -1 + 2q^{-1}\sum_{t \in T(f)} \sum_{a = 1}^q g^{-at} \cdot \prod_{i \in S(f)} g^{ax_i}\]
\end{lemma}
\begin{proof}
	For a fixed $t$, let $P_t(x) = \sum_{a = 1}^q g^{-at} \cdot \prod_{i \in S(f)} g^{ax_i}$.
	We show the following:
	\[P_t(x) = \begin{cases}
		q & \text{ if } \sum_{i=1}^n x_i = t,\\
		0 & \text{ otherwise.}
	\end{cases}\]

	First, notice that if $\sum_{i=1}^n x_i = t$, 
	then $g^{-at} \cdot \prod_{i \in S(f)} g^{ax_i} = g^{-at} g^{a \sum_{i=1}^n x_i} = 1$, 
	and therefore $P_t(x) = q$.

	Now, if $\sum_{i=1}^n x_i \neq t$, then $g^{-at} \cdot \prod_{i \in S(f)} g^{ax_i} = g^{ac}$ for some $c \neq 0 \mod q$.
	Therefore, we have:
	\begin{flalign*}
		P_t(x) 	&= \sum_{a=1}^q g^{ac}\\
				&= g^c\cdot(g^{qc} - 1)\cdot(g^c - 1)^{-1}\\
				&= 0 \text{ since $g$ has order $q$.}
	\end{flalign*}

	Therefore, $\sum_{t \in T(f)} \sum_{a = 1}^q g^{-at} \cdot \prod_{i \in S(f)} g^{ax_i} = \sum_{t \in T(f)} P_t$ is equal to $q$ if $\sum_{i=1}^n x_i \in T(f)$, and 0 otherwise.
	Multiplying by $2q^{-1}$ and substracting 1 yields the value of $f(x)$.
\end{proof}

\begin{lemma}
	Let \[P_C(X) = -|F| + 2q^{-1}\sum_{a = 1}^q \sum_{t = -n}^n g^{-at} \sum_{f \in F : t\in T(f)} \left(\prod_{i \in S^+(f)} X_i^a\right) \left(\prod_{i \in S^-(f)} X_i^{-a}\right)\]
	Then, for every $x\in\mbit^n$, we have:
	\[P_C(g^{x_1}, \ldots, g^{x_n}) = \sum_{f \in F} f(x).\]
\end{lemma}
\begin{proof}
	Use Lemma~\ref{lemma:symgate}, plug in $X_i = g^{x_i}$,
	take into account the fact that some inputs may be negated, 
	sum over $f\in F$, and reverse summation order.
\end{proof}


%%%%%%%%%%%%%%%%%%%%%%
\subsection{Moderate-size fields with small subgroups}
Evaluate over $\{g, g^{-1}\}^n$ in $\FF_p$, where $q\mid p-1$.
How to find $p,q$ so that we don't blow up the running time.


%%%%%%%%%%%%%%%%%%%%%%
\subsection{Fast multipoint evaluation of these symmetric polynomials}
In what follows, let $P_{a, t}$ denote the polynomial
\[P_{a, t}(X) = g^{-at}\sum_{f \in F : t\in T(f)} \left(\prod_{i \in S^+(f)} X_i^a\right) \left(\prod_{i \in S^-(f)} X_i^{-a}\right).\]
We use $P_{-a, t}$ to denote $P_{q-a, t}$: this is barely an abuse of notation
since all of our computation takes place with $X\in\gbit^n$ 
and $g$ has order $q$.

Notice that for every $X\in\gbit^n$, we have $P_{-a, t}(X) = g^{2at}P_{a, t}(X^-1)$,
where $X^{-1} = (X_i^{-1}; i = 1, \ldots, n)$. 
Therefore, for every $X\in\gbit^n$, we now rewrite $P_C$ as:
\begin{flalign*}
	P_C(X) 
	&= -|F| + 2q^{-1}\sum_{t = -n}^n \sum_{a=1}^q P_{a, t}(X)\\ 
	&= -|F| + q^{-1}\sum_{t = -n}^n \left(\sum_{a=1}^q P_{a, t}(X) + \sum_{a=0}^{q-1} P_{a, t}(X)\right)\\ 
	&= -|F| + q^{-1}\sum_{t = -n}^n \left(\sum_{a=1}^q P_{a, t}(X) + \sum_{a=1}^{q} P_{-a, t}(X)\right)\\ 
	&= -|F| + q^{-1}\sum_{t = -n}^n \sum_{a=1}^q P_{a, t}(X) + P_{-a, t}(X)\\
	&= -|F| + q^{-1}\sum_{t = -n}^n \sum_{a=1}^q P_{a, t}(X) + P_{-a, t}(X)
\end{flalign*}
To compute $P_C$ over all inputs in $\gbit^n$, it suffices to compute the values of $P_{a, t}$ over all inputs for every $a,t$. 
This only adds polynomial overhead since there are at most $2n+1$ values of $t$ and $q = O(n)$.

In what follows, fix a value of $t$, and let $P_a$ denote the polynomial $P_{a,t}$.
We give a recursive algorithm to evaluate $P_a + P_{-a}$ over all inputs.
At the $i$-th level of the recursion, we branch on the variable $X_i$.
The polynomial $P_a$ can be decomposed as a sum as follows:
\[P_a(X) = Q_0 + X_i^a Q_{a,+} + X_i^{-a} Q_{a,-}\]
Similarly, for $P_{-a}$:
\[P_{-a}(X) = Q_0 + X_i^{-a} Q_{-a,+} + X_i^{a} Q_{-a,-}\]
Here, again, we have $Q_{-a, +}(X) = g^{2a} Q_{a, +}(X^{-1})$, 
and $Q_{a, -}(X) = g^{-2a} Q_{-a, -}(X^{-1})$.
Therefore, we have:
\begin{flalign*}
	P_a(X) 		&= Q_0(X) + X_i^a Q_{a,+}(X) + X_i^{-a} Q_{a,-}(X^{-1})\\
	\text{and } 
	P_{-a}(X) 	&= Q_0(X) + X_i^{-a} g^{2a} Q_{a,+}(X^{-1}) + X_i^{a} g^{2a} Q_{a,-}(X^{-1})\\
\end{flalign*}
%%%%%%%%%%%%%%%%%%%%%%%%%
%%%%%%%%%%%%%%%%%%%%%%%%%
% \section{Evaluating weighted \ssym{} circuits}
% One way to deal with weighted circuits might be the following:

%%%%%%%%%%%%%%%%%%%%%%%%%
%%%%%%%%%%%%%%%%%%%%%%%%%
\bibliographystyle{plain}
\bibliography{biblio.bib}

\end{document}
