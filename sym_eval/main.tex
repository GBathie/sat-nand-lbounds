\documentclass[a4paper, 11pt]{article}
% \usepackage{ae,lmodern}
\usepackage[utf8]{inputenc}
\usepackage[T1]{fontenc}
\usepackage[USenglish]{babel}
\usepackage[margin=3cm]{geometry}
\usepackage{amsfonts}
\usepackage{amsmath}
\usepackage{amsthm}
\usepackage{amssymb}
\usepackage{todonotes}
\usepackage{enumitem}
\usepackage{algorithm}
\usepackage[noend]{algpseudocode}


\theoremstyle{plain}
\newtheorem{theorem}{Theorem}[section] % Number within sec ?
% \newtheorem{theorem}{Theorem}
\newtheorem{prop}[theorem]{Proposition}
\newtheorem{property}[theorem]{Property}
\newtheorem{lemma}[theorem]{Lemma}
\newtheorem{corollary}[theorem]{Corollary}
\newtheorem{proposition}[theorem]{Proposition}
\newtheorem{exercise}[theorem]{Exercise}
\newtheorem{conjecture}[theorem]{Conjecture}
\newtheorem{observation}[theorem]{Observation}
\theoremstyle{definition}
\newtheorem{definition}[theorem]{Definition}
\newtheorem{example}[theorem]{Example}
\theoremstyle{remark}
\newtheorem{remark}[theorem]{Remark}
% \newtheorem*{remark*}{Remark}
% \theoremstyle{definition}
% \newtheorem{problem}{Problem}
% \newtheorem{openproblem}{Open Problem}

\newcommand{\NN}{\mathbb{N}}%
\newcommand{\ZZ}{\mathbb{Z}}%
\newcommand{\Sbb}{\mathbb{S}}%
\newcommand{\Cc}{\mathcal{C}}%
\newcommand{\Gg}{\mathcal{G}}%
\newcommand{\Ff}{\mathcal{F}}%
\newcommand{\Ss}{\mathcal{S}}%
\newcommand{\eps}{\varepsilon}%
\newcommand{\epsm}{\eps^{-1}}%
\newcommand{\bit}{\{0,1\}}%
\DeclareMathOperator*{\poly}{poly}

\newcommand{\Omegat}[1]{\widetilde{\Omega}\left( #1 \right)}%
\newcommand{\Ot}[1]{\widetilde{O}\left( #1 \right)}%

\newcommand{\Poly}{\textsf{P}}%
\newcommand{\NP}{\textsf{NP}}%
\newcommand{\TIME}{\textsf{TIME}}%
\newcommand{\NTIME}{\textsf{NTIME}}%
\newcommand{\coNTIME}{\textsf{coNTIME}}%
\newcommand{\SAT}{\textsf{SAT}}%
\newcommand{\ParSAT}{\ensuremath\bigoplus\textsf{SAT}}%
\newcommand{\SharpSAT}{\#\textsf{SAT}}%
\newcommand{\NAND}{\textsf{NAND}}%
\newcommand{\ND}{\textsf{NDepth}}%
\newcommand{\NDL}[1]{\ND[ #1 \log n]}%
\newcommand{\ssym}{\textsf{SYM$\circ$SYM}}%

\newcommand{\todohere}{\todo[inline]{TODO}}%

%%%%%%%%%%%%%%%%%%%%%%
%%%%%%%%%%%%%%%%%%%%%%
\title{Fast multipoint evaluation for exponential size CNFs and SYM$\circ$SYM circuits}
\author{Gabriel Bathie, Ryan Williams}
\begin{document}
\maketitle

\begin{abstract}
We introduce a new way to encode size-$m$ \ssym{} circuits with possibly negated inputs
(which include CNFs and DNFs)
into polynomials that can be evaluated over all inputs in $\bit^n$ in time $O^*(2^n + m)$.
As a consequence, we obtain a $O^*(2^n)$-time algorithm that computes the
truthe table of $2^n$-size CNFs.
Lower bounds ensue. (do they though?)
\end{abstract}


%%%%%%%%%%%%%%%%%%%%%%
%%%%%%%%%%%%%%%%%%%%%%
\section{Introduction}

\subsection{Results overview}

\begin{theorem}
	For all $n, m$, \ssym{} circuits of size $m$ over $n$ variables can be evaluated
	on every input in time $O^*(2^n + m)$.
\end{theorem}

This yields a significant improvement over the exhaustive search algorithm, that takes time $O(2^nm)$.

\begin{corollary}
	For any $c < 4 \cos(\pi/7)$, \SAT{} does not have \NAND{} formulas of \textbf{size} $n^{c + o(1)}$.
\end{corollary}

\begin{theorem} (??)
	With our rules, we cannot do better than $4 \cos(\pi/7)$.
\end{theorem}

%%%%%%%%%%%%%%%%%%%%%%
%%%%%%%%%%%%%%%%%%%%%%
\section{Definitions}

A symmetric function is a boolean function whose output only depends on the number of 
its input that equal to 1.
More formally:
\begin{definition}[Symmetric functions]
	A boolean function $f: \bit^n \mapsto \bit$ is \textit{symmetric} if there exists a function
	$g: \{0,\ldots, n\} \mapsto \bit$ such that for every $x\bit^n$:
	\[f(x) = g(\sum_{i=1}^n x_i).\]
\end{definition}

\begin{definition}[\ssym{} circuits]
\end{definition}

wlog, \ssym{} circuit only contains so-called ``Exact-Majority'' gates.

%%%%%%%%%%%%%%%%%%%%%%
%%%%%%%%%%%%%%%%%%%%%%
\section{Proof of main result}


%%%%%%%%%%%%%%%%%%%%%%%%%
%%%%%%%%%%%%%%%%%%%%%%%%%
\bibliographystyle{plain}
\bibliography{biblio.bib}

\end{document}
